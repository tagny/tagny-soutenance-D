\section{Conclusions}
\subsection{Bilan}
\begin{frame}[c]{\mysubsectiontitle}	
	\begin{itemize} \small
		\item Définition de tâches importantes pour l'analyse de corpus de décisions 
		\begin{itemize} \scriptsize
			\item Formulation en problèmes d'analyse de données textuelles
			\item Production avec un expert de données annotées d'apprentissage 
		\end{itemize}
		\item Proposition et évaluation d'approches d'extraction de connaissances jurisprudentielles :
		\begin{itemize} \scriptsize
			\item Application du HMM  et CRF pour détecter les sections et les entités juridiques \cite{tagnyngompe2017neregc, tagnyngompe2019ner}
			\item Approche d'identification des demandes par catégorie basée sur la proximité entre des termes-clés appris et les sommes d'argent		
			\item Deux extensions du Gini-PLS pour identifier le sens du résultat (article en préparation pour \textit{\textbf{Stats}})
			\item Approche d'apprentissage d'une distance de similarité pour regrouper les décisions suivant les circonstances factuelles.	
		\end{itemize}
	   \item Démonstration d'applications en analyse descriptive sur un grand corpus de décisions	   
\end{itemize}
\end{frame}

\subsection{Perspectives}
\begin{frame}[c]{\mysubsectiontitle}

	\begin{itemize} \small
		\item Extensions des propositions 
		\begin{itemize}  \scriptsize
			\item Désambiguïsation les entités détectées pour indexer les décisions
			\item Expérimentation d'approches d'extraction des évènements et relations pour l'identification des demandes		
			\item Découverte des circonstances factuelles par  {modélisation thématique}
		\end{itemize}
		\item Applications
		\begin{itemize}  \scriptsize
			\item {Anonymisation des décisions} : confidentialité des informations
			\item {Analyse prédictive} : identifier les raisons qui poussent les juges à accepter une demande 
		\end{itemize}
	\end{itemize}
\end{frame}