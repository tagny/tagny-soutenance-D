\section{Conclusions}
\subsection{Bilan}
\begin{frame}[c]{Conclusions : bilan}
	\small
\begin{block}{Contributions}
	\begin{itemize}
		\item Etude de l'application du HMM  et CRF pour détecter les sections et les entités juridiques
		\item Approche d'identification des demandes basée sur la proximité entre les termes-clés et les sommes d'argent		
		\item Extensions du Gini-PLS pour identifier le sens du résultat
		\item Approche d'apprentissage d'une distance de similarité pour regrouper les décisions suivant les circonstances factuelles.
	\end{itemize}
\end{block}
\begin{alertblock}{Limites}
	\begin{itemize}
		\item Évaluation sur de faibles quantité de données annotées;
		\item Non expérimentation de méthodes récentes (réseaux de neurones)
	\end{itemize}
\end{alertblock}
\end{frame}

\subsection{Perspectives}
\begin{frame}[c]{Conclusions : perspectives}
	\small
	\begin{block}{Amélioration des propositions}
		\begin{itemize}
			\item \textbf{Désambiguïser les entités} détectées pour indexer les décisions
			\item Expérimentation des approches récentes pour l'identification des \textbf{demandes formalisées comme relation entre montant demandé et montant accordé}
			\item Découverte des circonstances factuelles vue comme \textbf{modélisation thématique}
		\end{itemize}
	\end{block}
	\begin{block}{Applications}
		\begin{itemize}
			\item \textbf{Anonymisation des décisions} : confidentialité des informations
			\item \textbf{Analyse prédictive} : identifier les raisons qui poussent les juges à accepter une demande
		\end{itemize}
	\end{block}
\end{frame}