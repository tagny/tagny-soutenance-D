\section{Conclusions}
\subsection{Bilan}
\begin{frame}[c]{\mysubsectiontitle}	
	\begin{itemize} \scriptsize
		\item Définition de problèmes importants d'analyse de corpus de décisions 
		\begin{itemize} \scriptsize
			\item Formulation en tâches de fouille de textes
			\item Production avec un expert de données annotées d'apprentissage
		\end{itemize}
		\item Proposition et évaluation d'approches d'extraction de connaissances jurisprudentielles :
		\begin{itemize} \scriptsize
			\item Application du HMM  et CRF pour détecter les sections et les entités juridiques
			\item Approche d'identification des demandes par catégorie basée sur la proximité entre des termes-clés appris et les sommes d'argent		
			\item Proposition et évaluation  d'extensions du Gini-PLS pour identifier le sens du résultat
			\item Approche d'apprentissage d'une distance de similarité pour regrouper les décisions suivant les circonstances factuelles.	
		\end{itemize}
	   \item Démonstration d'applications en analyse descriptive sur un grand corpus de décisions
	\end{itemize}
\end{frame}

\subsection{Perspectives}
\begin{frame}[c]{\mysubsectiontitle}

	\begin{itemize} \scriptsize
		\item Amélioration des propositions 
		\begin{itemize}  \scriptsize
			\item Désambiguïser les entités détectées pour indexer les décisions
			\item Expérimentation des approches récentes pour l'identification des \textbf{demandes formalisées comme relation entre montant demandé et montant accordé}
			\item Découverte des circonstances factuelles vue comme \textbf{modélisation thématique}
		\end{itemize}
		\item Applications
		\begin{itemize}  \scriptsize
			\item \textbf{Anonymisation des décisions} : confidentialité des informations
			\item \textbf{Analyse prédictive} : identifier les raisons qui poussent les juges à accepter une demande
		\end{itemize}
	\end{itemize}
\end{frame}