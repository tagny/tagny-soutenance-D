\section{Découverte des circonstances factuelles}
\subsection{Objectif de la tâche}
\begin{frame}[t]{\mysubsectiontitle}
	Déterminer les situations distinctes où sont formulées les demandes d'une catégorie données
	\begin{itemize} \scriptsize
		\item Exemple : catégorie "action en responsabilité civile professionnelle contre les avocats" (\textit{arcpa})
			\begin{itemize}\scriptsize
				\item cas $a$ : un avocat négligent qui envoie son assignation de manière tardive; 
				\item cas $b$ : un avocat qui n'a pas donné un conseil opportun, qui n'a pas soulevé le bon argument;
				\item cas $c$ : un avocat qui n'a pas rédigé un acte valide ou réussi à obtenir un avantage fiscal; 
				\item cas $d$ : un avocat attaqué par son adversaire et non par son propre client.
			\end{itemize}
		\item Formulation comme un problème de regroupement non supervisé des décisions
	\end{itemize}
\end{frame}
\subsection{Méthode}
\begin{frame}[t]{\mysubsectiontitle}
	Apprentissage et utilisation d'une distance basée sur la transformation d'un document en un autre
	\begin{itemize} \scriptsize
		\item Formulation de la distance pour un ensemble de modifications connues	
		\[{Dis_{M}}(d,d') = {f}({M}_{(d,d')}) = \frac{\sum\limits_{(d[k], d'[k]) \in {M}_{(d,d')}} Dis_{cos}(\overrightarrow{d[k]}, \overrightarrow{d'[k]})}{\vert d \vert}\] 
		\item Génération d'un corpus d'entraînement	$B_{M} = \lbrace ((d_1, d_2), {Dis}(d_1, d_2))_i\rbrace_{1 \leq i \leq \setsize{B_{M}}}$
		\item Entraînement d'un modèle de régression pour prédire la distance entre deux documents \[Dis_{M}(d_i, d_j) = Reg_{M}(\vec{d_{i}} - \vec{d_{j}})\]	
		\item Utilisation de la distance dans un algorithmes de regroupement (K-moyennes et K-medoides)
	\end{itemize}	
\end{frame}

\subsection{Sélection de la représentation des décisions}
\begin{frame}[t]{\mysubsectiontitle}
	Trouver la représentation qui discrimine les cas sur leur champ sémantique
	
	\begin{table}[ht]
		\scriptsize
		\begin{center}
			\begin{tabular}{|l|p{.85\textwidth}|}
				\hline
				\textbf{Corpus} & \textbf{Terminologie} \\ \hline
				\textit{arcpa} & chance, perte chance, avocat, perte, diligence, chance obtenir, perdre, client, devoir conseil, manquement
				\\ \hline
				\textit{cas a} & chance, perte chance, chance succès, perte, client, préjudice indemnisable, article code commerce, indemnisable, condamnation emporter, emporter nécessairement rejet
				\\ \hline
				\textit{cas b} & défense intérêt, intérêt client, avocat, contractuel égard, responsabilité contractuel droit, responsabilité professionnel avocat, contractuel droit commun, assurer défense intérêt, civil avocat, grief articuler
				\\ \hline
				\textit{cas c} & rédacteur acte, rédacteur, avocat rédacteur acte, avocat rédacteur, qualité rédacteur acte, rédaction acte, qualité rédacteur, projet acte, prendre initiative conseiller, initiative conseiller
				\\ \hline
				\textit{cas d} & revêtir aucun, revêtir aucun caractère, article code, article code procédure, faire référence aucun, fautif madame, civil profit autre, civil depuis, mention expresse, moyen dont
				\\ \hline
			\end{tabular}
		\end{center}
		\caption{Terminologies de  la catégorie  $arcpa$ et de ses cas}
	\end{table}
\end{frame}
\begin{frame}{Sélection de la représentation : résultats}
\begin{table}[!htb]
	\scriptsize
	\begin{center}
		\begin{tabular}[pos]{|l|c|p{0.7\textwidth}|}
			\hline
			\textbf{Distance}&\textbf{Base$^a$}&\textbf{Silhouette optimale   (pondération, réduction, dim.)} \\ \hline
			$Dis_{jaccard}$ & 0.001 & 0.212 (TP-NGL, FNM, 4) \\ \hline
			$Dis_{cos}$ & 0.002 & 0.202 (TP-NGL, FNM, 4) \\ \hline
			$Dis_{M}$ & -0.049 & 0.195 (TP-NGL, FNM, 4) \\ \hline
			$Dis_{braycurtis}$ & 0.002& 0.182 (TP-NGL, FNM, 4) \\ \hline
			$Dis_{euclidienne}$ & 0.001& 0.168  (TP-NGL, FNM, 4) \\ \hline
			$Dis_{manhattan}$ & -0.019& 0.17   (TP-NGL, FNM, 4) \\ \hline
			$Dis_{pearson}$ & 0.014 & 0.057 (TP-CHI2, aucune, 19763) \\ \hline
			$Dis_{wmd}$ & -0.096 &  - \\ \hline
		\end{tabular}				
	\end{center}
	
	$^a$ occurrence de mots pour $Dis_{wmd}$, et TF-IDF pour les autres distances.
	\caption{Meilleures représentations sur la catégorisation manuelle.} \label{tab:similarite:silhouette-vecteur-manuel}
\end{table}
\end{frame}

\subsection{Efficacité du regroupement}
\begin{frame}{Regroupement pour la catégorie annotée}
	\begin{table}[!htb]
		\centering \scriptsize
		\begin{tabular}[pos]{|l|l|c|c|c|c|c|c|c|}
			\hline
			{Distance}& {Algorithme}& {K}& {Silhouette}& {ARI} & {NMI} & {R} & {P} & ${F_1}$ \\ \hline
			$Dis_{M}$          & K-moyennes    & \textbf{3} & 0.438      & \textbf{0.407} & \textbf{0.423} & 0.552  & 0.654     & \textbf{0.599} \\ \hline
			$Dis_{M}$          & K-medoïdes  & 6 & 0.453      & 0.359 & 0.395 & 0.298  & 0.669     & 0.413 \\ \hline
			$Dis_{braycurtis}$ & K-moyennes    & 4 & 0.473      & 0.383 & 0.407 & 0.446  & 0.658     & 0.532 \\ \hline
			$Dis_{braycurtis}$ & K-medoïdes  & 5 & 0.448      & 0.344 & 0.375 & 0.331  & 0.645     & 0.437 \\ \hline
			$Dis_{cosine}$     & K-moyennes    & 4 & 0.528      & 0.383 & 0.407 & 0.446  & 0.658     & 0.532 \\ \hline
			$Dis_{cosine}$     & K-medoïdes  & 4 & 0.526      & \textbf{0.398} & \textbf{0.421} & 0.464  & 0.680     & \textbf{0.551} \\ \hline
			$Dis_{euclidean}$  & K-moyennes    & 5 & 0.478      & 0.365 & 0.395 & 0.341  & 0.670     & 0.452 \\ \hline
			$Dis_{euclidean}$  & K-medoïdes  & 5 & 0.456      & 0.313 & 0.346 & 0.335  & 0.619     & 0.434 \\ \hline
			$Dis_{jaccard}$    & K-moyennes    & 4 & 0.570      & 0.367 & 0.391 & 0.439  & 0.643     & 0.522 \\ \hline
			$Dis_{jaccard}$    & K-medoïdes  & 4 & \textbf{0.560}      & 0.389 & 0.412 & 0.451  & 0.666     & 0.538 \\ \hline
			$Dis_{manhattan}$  & K-moyennes    & 4 & 0.482      & 0.376 & 0.400 & 0.452  & 0.657     & 0.535 \\ \hline
			$Dis_{manhattan}$  & K-medoïdes  & 5 & 0.452      & 0.368 & 0.397 & 0.345  & 0.675     & 0.456 \\ \hline
			$Dis_{pearson}$    & K-moyennes    & 2 & \textbf{0.611}      & 0.054 & 0.072 & 0.746  & 0.453     & 0.564 \\ \hline
			$Dis_{pearson}$    & K-medoïdes  & 2 & 0.171      & 0.152 & 0.166 & 0.598  & 0.482     & 0.534 \\ \hline
			$Dis_{wmd}$      & K-medoïdes  & 2 & 0.332      & -0.016 & 0.002 & 0.545  & 0.397     & 0.459 \\ \hline
		\end{tabular}
		\caption{\scriptsize Evaluation de la catégorisation par K-moyennes et K-medoïdes sur ${D}_{arcpa}$ avec détermination du nombre de clusters basée sur la silhouette.} \label{tab:similarite:validation-supervisee-optKbySilhouette}
	\end{table}
\end{frame}


\begin{frame}[c]{Regroupement des catégories non annotées}
	\newlength{\mrcell}
	\setlength{\mrcell}{0.8cm}
	\begin{table}[!htb]
		\tiny
		\begin{center}						
				\begin{tabular}[pos]{|l|l|l|c|c|}
					\hline				
					\multirow{6}{\mrcell}{${D}_{doris}$ (59)}  & $Dis_{M}$ & K-medoïdes & 2 & 0.509  \\ \cline{2-5}
					& $Dis_{M}$ & K-moyennes & 3 & 0.527  \\ \cline{2-5}
					& $Dis_{cosine}$ & K-medoïdes & 5 & 0.549 \\ \cline{2-5}
					& $Dis_{cosine}$ & K-moyennes & 4 & 0.586 \\ \cline{2-5}
					& $Dis_{jaccard}$ & K-medoïdes & 3 & 0.600 \\ \cline{2-5}
					& $Dis_{jaccard}$ & K-moyennes & 4 & 0.645 
					\\ \hline
				\end{tabular}
		\end{center}						
		\caption{\tiny Evaluation non-supervisée des K-moyennes et K-medoïdes sur ${D}_{doris}$.} \label{tab:similarite:validation-nonsupervisee}
	\end{table}
	\begin{table}[ht]
		\centering \tiny
		\begin{tabular}{|c|p{.85\textwidth}|}
			\hline
			{Cluster} & \multicolumn{1}{c|}{Terminologie ($ngl$)} \\ \hline
			\textit{0} & excéder inconvénient, inconvénient normal, excéder inconvénient normal, normal voisinage, inconvénient normal voisinage, inconvénient, trouble excéder inconvénient, trouble excéder, excéder, normal
			\\ \hline
			\textit{1} & copropriétaire, syndicat copropriétaire, syndicat, condamner in, anormal voisinage, trouble anormal voisinage, in, trouble anormal, syndic, jouissance subir
			\\ \hline
			\textit{2} & deux fond|fonds, séparatif deux fond|fonds, limite séparatif deux, ordonner démolition, séparatif deux, implanter, condamner démolir, devoir établir toit, devoir établir, toit manière
			\\ \hline
			\textit{3} & manière plus, chose manière plus, chose manière, usage prohiber loi, prohiber loi règlement, prohiber loi, absolu, usage prohiber, manière plus absolu, plus absolu
			\\ \hline
			\textit{4} & situer zone, hauteur @card@ mètre, hauteur dépasser, appel contester, vitrer, dont hauteur dépasser, urbaniser, recevabilité <unknown> appel, cahier charge lotissement, charge lotissement
			\\ \hline
		\end{tabular}
		
		%\textit{10 premiers termes de 1 à 3 mots sélectionnés à l'aide du coefficient de corrélation $ngl$ }
		\caption{\tiny Terminologies des circonstances factuelles découvertes en combinant les K-medoïdes et la distance cosinus sur ${D}_{doris}$.}\label{tab:similarite:terminologie-clusters-doris}
	\end{table}
\end{frame}

\begin{frame}{Résumé}
	\begin{enumerate}
		\item Formulation comme problème de regroupement non supervisé de décisions de la catégorie
		\item Méthode d'apprentissage d'une distance de dis-similarité au sein d'une catégorie
		\item Sélection de la représentation des textes qui reflète la notion subjective de similarité de l'expert
		\item Expérimentation des propositions sur 7 catégories de demandes dont 1 annotées
	\end{enumerate}
\end{frame}
