\begin{frame}[c]{Appliquer le HMM ou le CRF pour l'annotation}
	\only<1>{1. Méthodes : 	Modèles probabilistes à états et observations

\scriptsize
\begin{table}[]%width=\linewidth
	%\begin{tabular}[]{p{0.40\linewidth}|p{0.55\linewidth}}
	\begin{tabular}[]{c|c}
		\toprule
{\textbf{HMM}} & {\textbf{CRF}} \\
%		\midrule
%		\textbf{Generative} models 	& \multicolumn{2}{c}{"\textbf{Discriminative}" or "\textbf{Conditional}" models } \\[0.25em]
%\midrule		
%"\textbf{generate}" input	& {"\textbf{condition}" on input }\\%[0.25em]
\midrule
{un seul descripteur  par observation}	& {plusieurs descripteurs complexes par observation}\\%[0.25em]
\midrule	
		\begin{tikzpicture}[->,>=stealth',shorten >=1pt,auto,node distance=1.3cm,
                    semithick]
  \node[state] (S1)                    {$s_{t-1}$};
  \node[state]         (S2) [right of=S1] 	  {$s_{t}$};
  \node[state]         (O) [below of=S2] {$o_{t}$};
  \path (S1) edge              node {} (S2)
        (S2) edge              node {} (O);
\end{tikzpicture}
				& 

\begin{tikzpicture}[auto,>=stealth',shorten >=1pt,auto,node distance=1.3cm,
                    semithick]
  \node[state] (S1)                    {$s_{t-1}$};
  \node[state]         (S2) [right of=S1] 	  {$s_{t}$};
  \node[state]         (O) [below of=S2] {$o_{t}$};
  \path (S1) edge              node {} (S2)
        (S2) edge              node {} (O);
\end{tikzpicture}					
					\\%[0.25em]
\midrule
$P_\lambda(S|O) = \prod\limits_{t=1}^{T} P(s_t \vert s_{t-1}) * P(o_t \vert s_{t})$  & $P_\lambda(S|O) = \frac{1}{Z(O)}exp\left( \sum\limits_{t=1}^{T}\sum\limits_{k} \lambda_k f_k(s_{t-1},s_t, o_t) \right) $ \\
% & & & \\
\tiny \cite{Seymore1999hmm} & \tiny \cite{peng2006crf} \\ 
		\bottomrule
	\end{tabular}
\end{table}

\footnotesize

Objectif: Trouver la séquence la plus probable d'étiquetage pour l'ensemble du texte

\textbf{Entrainement fait sur des séquences préalablement étiquetées}
}
	\only<2>{2. Données}
	\only<3>{3. Résultats}	
\end{frame}