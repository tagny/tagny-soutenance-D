	%-=-=-=-=-=-=-=-=-=-=-=-=-=-=-=-=-=-=-=-=-=-=-=-=
%
%        LOADING DOCUMENT
%
%-=-=-=-=-=-=-=-=-=-=-=-=-=-=-=-=-=-=-=-=-=-=-=-=

\documentclass[newPxFont,pagenumber]{beamer}
\usetheme{sthlm}
%\usecolortheme{sthlmv42}

%-=-=-=-=-=-=-=-=-=-=-=-=-=-=-=-=-=-=-=-=-=-=-=-=
%        LOADING PACKAGES
%-=-=-=-=-=-=-=-=-=-=-=-=-=-=-=-=-=-=-=-=-=-=-=-=
\usepackage[utf8]{inputenc}
\usepackage[frenchb]{babel}
\usepackage[normalem]{ulem}
\usepackage{caption}
\captionsetup{font=scriptsize}
%\usepackage[font=footnotesize]{subcaption}
% in preamble
\usepackage{chronology}
\usepackage{pgf}
\usepackage{tikz}
\usetikzlibrary{arrows,automata}
\usepackage{array,multirow}
\usepackage{nameref}
\makeatletter
\newcommand*{\currentname}{\@currentlabelname}
\makeatother

\graphicspath{ {fig/} }

\usepackage[linesnumbered,ruled,vlined]{algorithm2e}
% add page number
%\usepackage[defaultsans]{cantarell}

\newcommand{\p}{\mathbb{P}}

\setbeamerfont{title}{series=\upshape}
\setbeamertemplate{footline}{\hfill\footnotesize\insertframenumber\hskip3pt\null\vskip3pt}

\newcommand{\argmax}{\mathop{\mathrm{argmax}}\limits}
\renewcommand{\max}{\mathop{\mathrm{max}}\limits}

\renewcommand{\event}[3][e]{%
  \pgfmathsetlength\xstop{(#2-\theyearstart)*\unit}%
  \ifx #1e%
    \draw[fill=black,draw=none,opacity=0.5]%
      (\xstop, 0) circle (.2\unit)%
      node[opacity=1,rotate=45,right=.2\unit] {#3};%
  \else%
    \pgfmathsetlength\xstart{(#1-\theyearstart)*\unit}%
    \draw[fill=black,draw=none,opacity=0.5,rounded corners=.1\unit]%
      (\xstart,-.1\unit) rectangle%
      node[opacity=1,rotate=45,right=.2\unit] {#3} (\xstop,.1\unit);%
  \fi}%

\addto\captionsfrench{%
\renewcommand{\figurename}{\scriptsize {\scshape Figure}}
\renewcommand{\tablename}{\scriptsize {\scshape Table}}
}

%-=-=-=-=-=-=-=-=-=-=-=-=-=-=-=-=-=-=-=-=-=-=-=-=
%        BEAMER OPTIONS
%-=-=-=-=-=-=-=-=-=-=-=-=-=-=-=-=-=-=-=-=-=-=-=-=

%\setbeameroption{show notes}

%-=-=-=-=-=-=-=-=-=-=-=-=-=-=-=-=-=-=-=-=-=-=-=-=
%
%	PRESENTATION INFORMATION
%
%-=-=-=-=-=-=-=-=-=-=-=-=-=-=-=-=-=-=-=-=-=-=-=-=

\title{\vspace{0.4cm}\small Méthodes D'Analyse Sémantique De Corpus De Décisions Jurisprudentielles}
\subtitle{\scriptsize Soutenance de thèse de doctorat en informatique de l'IMT Mines Alès}
%\date{\small{\jobname}}
%\date{\scriptsize Début de thèse: 15 Décembre 2015}
\date{\scriptsize 24 janvier 2020}
\author{\small Gildas Tagny Ngompé}
\institute{\tiny \textbf{Jury:} \begin{itemize}
\item Stéphane MUSSARD, Professeur, Université de Nîmes (Directeur de thèse)
\item Jacky MONTMAIN, Professeur, IMT Mines Alès (Co-directeur de thèse)
\item Sandra BRINGAY, Professeur, Université Paul Valéry Montpellier (Rapporteur)
\item Boughanem MOHAND, Professeur, Université Toulouse III Paul Sabatier (Rapporteur)
\item Françoise SEYTE, Maître de Conférences (HDR), Université de Montpellier (Examinateur)
\item Fabrice MUHLENBACH,  Maître de Conférences, Université Jean Monnet de Saint-Étienne (Examinateur)
\item Guillaume ZAMBRANO, Maître de Conférences, Université de Nîmes (Encadrant de proximité)
\item Sébastien HARISPE,  Maître Assistant, IMT Mines Alès (Encadrant de proximité)
\end{itemize}}

\hypersetup{
pdfauthor = {\author{}: tagnyngompe@gmail.com},
pdfsubject = {},
pdfkeywords = {},
pdfmoddate= {D:\pdfdate},
pdfcreator = {}
}

\begin{document}
\nocite{}
%-=-=-=-=-=-=-=-=-=-=-=-=-=-=-=-=-=-=-=-=-=-=-=-=
%
%	TITLE PAGE
%
%-=-=-=-=-=-=-=-=-=-=-=-=-=-=-=-=-=-=-=-=-=-=-=-=
\begin{frame}[plain]
	\titlepage
\end{frame}
%}
%-=-=-=-=-=-=-=-=-=-=-=-=-=-=-=-=-=-=-=-=-=-=-=-=
%
%	TABLE OF CONTENTS: Plan
%
%-=-=-=-=-=-=-=-=-=-=-=-=-=-=-=-=-=-=-=-=-=-=-=-=
\section*{Plan}
\begin{frame}[c]{\currentname}
\tableofcontents[hideallsubsections]
\end{frame}

\section*{Introduction}

\begin{frame}[c]{Contexte}
	
\end{frame}


\section*{Bibliographie}

\begin{frame}[c]{Annotation dans les décisions judiciaires}
	
\end{frame}


\section*{Méthodes}
\begin{frame}[c]{Étiquetage hmm et crf}
	
\end{frame}


\begin{frame}[c]{extraction des demandes}
	
\end{frame}

\begin{frame}[c]{découverte des circonstances factuelles}
	
\end{frame}

\section*{Résultats}

\section*{Discussions}

\section*{Conclusions}


%\section{Motivations et objectifs}
%
%\begin{frame}[c]{Les juristes analysent les décisions afin d'anticiper}
%\includegraphics[width=\textwidth]{lawyerwork.png}
%\end{frame}
%
%\begin{frame}[c]{Défis: grand volume de décisions}
%\textbf{Plus de 4 millions de décisions prononcées / an}
%\begin{table}[!htb]
%{
%\footnotesize
%\begin{center}
%\begin{tabular}{|p{2cm}|c|c|c|c|c|}
%\hline
% & \textbf{2010} & \textbf{2011} & \textbf{2012} & \textbf{2013} & \textbf{2014} \\
% \hline
% \textbf{Justice civile} & 2 673 131  & 2 654 179 & 2 647 813 & 2 761 554  & 2 618 374 \\
% \hline
%\textbf{Justice pénale} & 1 173 242 & 1 180 586 & 1 251 979 & 1 303 469 & 1 203 339 \\
% \hline
% \textbf{Justice administrative} & 224 787 & 225 608 & 228 680 & 221 882 & 230 477 \\
% \hline
%\end{tabular}
%\textit{\tiny{Source: \url{http://www.justice.gouv.fr/budget-et-statistiques-10054/chiffres-cles-de-la-justice-10303/}}}  
%\end{center}
%}
%\caption{Nombre de décisions prononcées en France par an}\label{decisionstats}
%\end{table}
%\end{frame}
%
%\begin{frame}[t]{Défis: Recherches et analyses sémantiques difficiles}
%
%Moteurs de recherche juridique à mots-clés 
%
%Pas d'analyse synthétique des décisions 
%
%%\begin{figure}
%\fbox{\includegraphics[width=0.9\paperwidth]{jurica.png}}
%
%\textit{\tiny{Source: \url{LexisNexis.com}}} 
%%\caption{Formulaire de recherche}
%%\end{figure}
%\end{frame}
%
%
%\begin{frame}{Défis: Documents non-structurés}
%\scriptsize
%\begin{columns}
%\begin{column}{.50\linewidth}
%ARRÊT N°
%
%R.G: 11/03924
%
%...
%
%{COUR D'APPEL} DE {NÎMES}
%
%{CHAMBRE CIVILE}
%
%{1ère Chambre A}
%
%ARRÊT DU {20 MARS 2012}
%
%APPELANTE:
%
%{Madame Michéle A.} ...
%
%assistée de la {SELARL VAJOU}, ...
%
%INTIMES:
%
%{Monsieur Martial B} ...
%
%assisté de la {SCP MARION GUIZARD PATRICIA SERVAIS}, ...
%
%COMPOSITION DE LA COUR LORS DU DÉLIBÉRÉ:
%
%{M. Dominique BRUZY, Président}
%
%{M. Serge BERTHET, Conseiller}
%
%...
%\end{column}
%\begin{column}{.50\linewidth}
%FAITS, PROCEDURE, ...
%
%Madame Michèle A. demande:
%
%...
%
%- de condamner Madame JONES-B. à lui payer la somme de {2.500 euros} au titre de l'{article 700 du Code de Procédure Civile}, 
%
%\vspace{0.4cm}
%
%PAR CES MOTIFS, LA COUR:
%
%...
%
%Vu l'{article 809 du Code de Procédure Civile},
%
%...
%
%{Déboute Madame A. de sa demande de provision sur dommages-intérêts.}
%
%...
%
%Vu l'{article 700 du Code de Procédure Civile},
%
%Condamne Madame JONES-B. à verser à Madame A. la somme de {2.500 euros}.
%\end{column}
%\end{columns}
%\end{frame}
%
%\begin{frame}{Notre projet: Automatiser la structuration et l'analyse}
%\includegraphics[width=\textwidth]{pipeline-cassandra2.png}
%
%Elaboration et mise en oeuvre de techniques de :
%\begin{itemize}
%\item Traitement du langage naturel 
%
%\item Représentation des connaissances
% 
%\item Recherche d'information
%\end{itemize}
%
%\end{frame}
%
%
%%%-=-=-=-=-=-=-=-=-=-=-=-=-=-=-=-=-=-=-=-=-=-=-=-=
%%%
%%%	Questions
%%%
%%%-=-=-=-=-=-=-=-=-=-=-=-=-=-=-=-=-=-=-=-=-=-=-=-=
%\section{Détection de sections et d'entités}
%
%
%\begin{frame}{Sectionner les décisions pour organiser l'extraction}
%\scriptsize
%\begin{columns}
%\begin{column}{.45\linewidth}
%\fbox{\begin{minipage}{\textwidth}ARRÊT N°
%
%R.G: \textcolor{red}{11/03924}
%
%\textcolor{red}{COUR D'APPEL} DE \textcolor{red}{NÎMES}
%
%\textcolor{red}{CHAMBRE CIVILE}
%
%\textcolor{red}{1ère Chambre A}
%
%ARRÊT DU \textcolor{red}{20 MARS 2012}
%
%APPELANTE:
%
%\textcolor{red}{Madame Michéle A.} ...
%
%assistée de la \textcolor{red}{SELARL VAJOU}, ...
%
%INTIMES:
%
%\textcolor{red}{Monsieur Martial B} ...
%
%assisté de la \textcolor{red}{SCP MARION GUIZARD PATRICIA SERVAIS}, ...
%
%COMPOSITION DE LA COUR LORS DU DÉLIBÉRÉ:
%
%\textcolor{red}{M. Dominique BRUZY, Président}
%
%\textcolor{red}{M. Serge BERTHET, Conseiller}
%
%...
%\end{minipage}}
%\vspace{0.1cm}
%
%{\normalsize \textbf{Entêtes}: méta-données}
%\end{column}
%\begin{column}{.55\linewidth}
%\fbox{\begin{minipage}{\textwidth}FAITS, PROCEDURE, ...
%
%Madame Michèle A. demande:
%
%...
%
%- de condamner Madame JONES-B. à lui payer la somme de \textcolor{red}{2.500 euros} au titre de l'\textcolor{red}{article 700 du Code de Procédure Civile}, 
%\end{minipage}}
%\vspace{0.1cm}
%
%{\normalsize \textbf{Corps}: demandes, arguments et normes }
%
%\vspace{0.4cm}
%
%\fbox{\begin{minipage}{\textwidth}PAR CES MOTIFS, LA COUR:
%
%...
%
%Vu l'\textcolor{red}{article 809 du Code de Procédure Civile},
%
%...
%
%\textcolor{red}{Déboute Madame A. de sa demande de provision sur dommages-intérêts.}
%
%...
%
%Vu l'\textcolor{red}{article 700 du Code de Procédure Civile},
%
%Condamne Madame JONES-B. à verser à Madame A. la somme de \textcolor{red}{2.500 euros}.
%\end{minipage}}
%\vspace{0.1cm}
%
%{\normalsize \textbf{Dispositif}: résultats et normes}
%
%\end{column}
%\end{columns}
%\end{frame}
%
%\begin{frame}{Entités et sections à détecter}
%
%\scriptsize
%\begin{table}[!htb]
%\centering
%%\scriptsize
%\begin{tabular}[c]{|l|c|p{0.6\textwidth}|}
%\hline
%\textbf{Entités} & \textbf{Labels} & \textbf{Exemples}\\\hline
%\multicolumn{3}{|c|}{\textbf{Section entête (E)}} \\
%\hline
%Numéro R.G. & \textbf{RG} & "10/02324", "60/JAF/09" \\ \hline
%Ville & \textbf{VL}& "NÎMES", "Agen", "Toulouse" \\ \hline
%Type de juridiction & \textbf{JR} & "COUR D'APPEL" \\ \hline
%Formation & \textbf{FM} & "1re chambre", "Chambre économique" \\ \hline
%Date & \textbf{DT} & "01 MARS 2012", "15/04/2014"\\ \hline
%Partie appelante & \textbf{AP} & "SARL K.", "Syndicat ...", "Mme X ..."\\ \hline
%Partie intimée & \textbf{IM} & - // - \\ \hline
%Partie intervenante & \textbf{IV} & - // - \\ \hline
%Avocat & \textbf{AV} & "Me Dominique A., avocat au barreau de Papeete"\\ \hline
%Juge & \textbf{JG} & "Monsieur André R.", "Mme BOUSQUEL" \\ \hline
%fonction du juge & \textbf{FT} & "Conseiller", "Président"\\ \hline
%\multicolumn{3}{|c|}{\textbf{Corps (T) et dispositif (D)}} \\ \hline
%
%Norme & \textbf{NO} & "l' article 700 NCPC", "articles 901 et 903"  \\ \hline
%\hline
%Élément à éviter & \textbf{O} & \textit{tout élément ne faisant partie d'aucune entité ciblée} \\ \hline
%\end{tabular} 
%
%\caption{Entités et leurs labels par section.}\label{relevantinfo}
%\end{table}
%\end{frame}
%
%\begin{frame}{Architecture proposée}
%\includegraphics [width=\textwidth]{archAppli.png}
%\end{frame}
%
%
%\begin{frame}{Approches probabilistes d'étiquetage de séquence}
%Modèles probabilistes à états et observations
%
%\scriptsize
%\begin{table}[]%width=\linewidth
%	%\begin{tabular}[]{p{0.40\linewidth}|p{0.55\linewidth}}
%	\begin{tabular}[]{c|c}
%		\toprule
%{\textbf{HMM}} & {\textbf{CRF}} \\
%%		\midrule
%%		\textbf{Generative} models 	& \multicolumn{2}{c}{"\textbf{Discriminative}" or "\textbf{Conditional}" models } \\[0.25em]
%%\midrule		
%%"\textbf{generate}" input	& {"\textbf{condition}" on input }\\%[0.25em]
%\midrule
%{un seul descripteur  par observation}	& {plusieurs descripteurs complexes par observation}\\%[0.25em]
%\midrule	
%		\begin{tikzpicture}[->,>=stealth',shorten >=1pt,auto,node distance=1.3cm,
%                    semithick]
%  \node[state] (S1)                    {$s_{t-1}$};
%  \node[state]         (S2) [right of=S1] 	  {$s_{t}$};
%  \node[state]         (O) [below of=S2] {$o_{t}$};
%  \path (S1) edge              node {} (S2)
%        (S2) edge              node {} (O);
%\end{tikzpicture}
%				& 
%
%\begin{tikzpicture}[auto,>=stealth',shorten >=1pt,auto,node distance=1.3cm,
%                    semithick]
%  \node[state] (S1)                    {$s_{t-1}$};
%  \node[state]         (S2) [right of=S1] 	  {$s_{t}$};
%  \node[state]         (O) [below of=S2] {$o_{t}$};
%  \path (S1) edge              node {} (S2)
%        (S2) edge              node {} (O);
%\end{tikzpicture}					
%					\\%[0.25em]
%\midrule
%$P_\lambda(S,O) = \prod\limits_{t=1}^{T} P(s_t \vert s_{t-1}) * P(o_t \vert s_{t})$  & $P_\lambda(S|O) = \frac{1}{Z(O)}exp\left( \sum\limits_{t=1}^{T}\sum\limits_{k} \lambda_k f_k(s_{t-1},s_t, o_t) \right) $ \\
%% & & & \\
%\tiny \cite{Seymore1999hmm} & \tiny \cite{peng2006crf} \\ 
%		\bottomrule
%	\end{tabular}
%\end{table}
%
%\normalsize
%
%Objectif: Trouver la séquence la plus probable d'étiquetage pour l'ensemble du texte
%
%\textbf{Entrainement sur des séquences préalablement étiquetées}
%\end{frame}
%
%\begin{frame}[c]{Premiers résultats \cite{tagny2017sectNerhmmcrf}}
%\begin{table}[!htb]
%\scriptsize
%
%\centering
%\begin{tabular}{|c|c|c|c|c|c|c|c|c|c|}
%\hline
% & \multicolumn{3}{c|}{HMM}  & \multicolumn{3}{c|}{CRF-}  & \multicolumn{3}{c|}{CRF+} \\
%\hline
%\textit{labels} & P & R & F1 & P & R & F1 & P & R & F1 \\
%\hline
% \multicolumn{10}{|c|}{\textit{\textit{Section Entête (E)}}} \\
%\hline
% AP & 35.3 &  14.1 & 20.1  & 64.9 & 48.8 & 55.6 & 92.0 & 86.7 & 89.3 \\
%\hline
% AV & 83.8 &  98.3 & 90.5  & 96.4 & 97.5 & 96.9 & 97.6 & 98.1 & 97.9 \\
% \hline
% DT & 70.9 & 72.6 & 71.7  & 94.4 & 86.8 & 90.4 & 98.8 & 97.7 & 98.2 \\
% \hline
%FM & 87.6 &  93.7 & 90.5  & 98.8 & 98.4 & 98.6 & 98.9 & 99.3 & 99.1 \\
% \hline
%FT &  88.8 & 59.8 & 71.3  & 94.2 & 92.3 & 93.3 & 97.1 & 95.5 & 96.3 \\
% \hline
%IM  & 53.1 & 57.4 & 55.1  & 67.2 & 64.6 & 65.8 &  89.3 & 88.1 & 88.7  \\
% \hline
% \textcolor{red}{IV} & - & 2.2 & - & 25.9 & 26.5 & 26.2 & 67.3 & 41.4 & \textcolor{red}{46.4} \\
% \hline
%JG  & 68.0 & 85.7 & 75.7  & 96.2 & 95.7 & 96.0 & 98.1 & 97.7 & 97.9 \\
% \hline
%JR  & 75.8 & 99.5 & 86.0  & 98.6 & 99.4 & 99.0 & 99.3 & 99.4 & 99.4 \\
% \hline
%RG  &  - & 0  & - & 83.7 & 46.1 & 59.4 & 98.6 & 97.4 & 98.0 \\
%\hline
%VL & 93.1 & 27.9 & 42.6  & 98.2 & 98.4 & 98.3 & 99.0 & 99.0 & 99.0 \\
%\hline
% \multicolumn{10}{|c|}{\textit{\textit{Sections inférieures (T \& D )}}} \\
% \hline
%NO & 92.9 & 90.9 & 91.9 & 96.0 & 93.8 & 94.9 & 97.9 & 96.5 & 97.2\\
%\hline
%\end{tabular}
%\caption{Précision (P), rappel (R), F1-mesure (F1) au niveau des mots ($\%$).}\label{prf-entity}
%\end{table}
%\end{frame}
%
%\begin{frame}{Premiers travaux \cite{tagny2017sectNerhmmcrf}}
%\begin{itemize}
%\item Utilité de la prise en compte des particularités des textes
%\begin{itemize}
%\item forme : le mot est-il en majuscule, lemmes, longueur de la ligne, ...
%\item contexte : mots voisins, position par rapport à un mot-clé, ...
%\end{itemize}
%\item Certaines entités restent difficiles à détecter
%\end{itemize}
%
%\vspace{1.5cm}
%
%\textbf{\Large \textit{Comment améliorer les résultats ?}}
%
%Définir plus de caractéristiques:
%
%\begin{itemize}
%\item 14 pour les sections
%\item 35 pour les entêtes
%\item 28 pour les normes
%\end{itemize}
%\end{frame}
%%\begin{table}[!htb]
%%\scriptsize
%%\centering
%%\begin{tabular}{|c|c|c|c|c|c|c|c|c|c|}
%%\hline
%% & \multicolumn{3}{c|}{HMM} & \multicolumn{3}{c|}{CRF-} & \multicolumn{3}{c|}{CRF+} \\
%%\hline
%% & P & R & F1  & P & R & F1 & P & R & F1 \\
%%\hline
%%Entête & 84.2 & 91.8 & 87.8  & 93.8 & 85.4 & 89.3 & 99.3 & 99.6 & 99.5 \\
%%\hline
%%Corps & 88.4 & 63.9 & 74.1  & 86.3 & 98.2 & 91.8 & 99.8 & 99.5 & 99.7\\
%%\hline
%%Dispositif & 15.4 & 47.0 & 23.0  & 100.0 & 8.5 & 15.6 & 98.0 & 100.0 & 98.9 \\
%%\hline
%%\textit{Moyenne} & 62.7 & 67.6 & 67.6  & 93.3 & 64.0 & 64.0 & 99.7 & 99.8 & 99.8  \\
%%\hline
%%\end{tabular}
%%\caption{Précision (P), rappel (R), F1-mesure (F1) au niveau des lignes ($\%$).}
%%\label{prf-zoning}
%%\end{table}
%
%\begin{frame}{Résultats avec plus de caractéristiques}
%\begin{table}
%\scriptsize
%\begin{tabular}{|l|ccc|}
%\hline
%         & Precision &  Recall  & F1 \\\hline
%I-corps &   99.57\% &  99.69\% &  99.63 \\
%I-dispositif &   98.63\% &  97.59\% &  98.11 \\
%I-entete &   99.51\% &  99.55\% &  99.53 \\\hline
%Overall &   99.48\% &  99.48\% &  99.48 \\\hline
% \noalign{\smallskip}\hline\noalign{\smallskip}
%I-appelant &   84.34\% &  76.27\% &  80.10 \\
%I-avocat &   98.02\% &  98.15\% &  98.09 \\
%I-date  &   98.00\% &  96.60\% &  97.30 \\
%I-fonction &   95.23\% &  95.13\% &  95.18 \\
%I-formation &   98.80\% &  99.45\% &  99.12 \\
%I-intervenant &   83.38\% &  68.26\% &  \textcolor{red}{75.07} \\
%I-intime &   82.54\% &  83.33\% &  82.93 \\
%I-juge  &   97.55\% &  97.23\% &  97.39 \\
%I-juridiction &   98.91\% &  99.69\% &  99.30 \\
%I-rg    &   97.81\% &  97.44\% &  97.62 \\
%I-ville &   98.94\% &  99.15\% &  99.04 \\\hline
%Overall &   95.13\% &  94.51\% &  94.82 \\\hline
% \noalign{\smallskip}\hline\noalign{\smallskip}
%I-norme &   97.14\% &  96.09\% &  96.62 \\\hline
%\end{tabular}
%\caption{Résultats du CRF avec l'ajout de caractéristiques.}
%\end{table}
%\end{frame}
%
%
%\begin{frame}{Sélection des caractéristiques}
%\begin{table}[!h]
%\tiny
%\begin{tabular}{l|c|ccc}
%\hline\noalign{\smallskip}
%Detection Task & Tagger & {Token-level F1} & {Entity-level F1}& Features subset \\ \hline
%%\noalign{\smallskip}\svhline\noalign{\smallskip}
%\multirow{7}{*}{Sections} 		& \multirow{4}{*}{CRF} & 99.31 & 90.48 & BDS  \\
%  				&  & \textbf{99.55} & 85.76 & \textbf{SFFS} \\
%                &  & 99.46 & 90.03 & ALL \\
%                &  & 91.75 & 60.26 & token \\  \cline{2-5}
%                 & \multirow{3}{*}{HMM} & \textbf{90.99} & 3.89 & \textbf{absLength} \\ 
% & & 86.97 & 3.65 & relLength \\   
%  &  & 37.59 & 18.81 & token \\ \hline
%\multirow{7}{*}{Header entities}	& \multirow{4}{*}{CRF} & 92.69 & 90.47 & BDS  \\
%				&  & \textbf{93.00} & 90.76 & \textbf{SFFS}  \\ 
%                &  & 92.74 & 90.81 & ALL \\
%                &  & 82.73 & 72.17 & token \\ \cline{2-5}
%                  &  \multirow{3}{*}{HMM}  & \textbf{78.61} & \textbf{56.93} &  \textbf{token} \\ 
%  &    & 68.04 & 32.96 &  lemma\_W0 \\ 
%  &    & 38.54 & 7.95 &  POS \\ \hline
%\multirow{6}{*}{Norms} 			& \multirow{4}{*}{CRF} & \textbf{96.31} & 90.80 & \textbf{BDS} \\ 
%				&  & 95.57 & 89.29 & SFFS \\ 
%                &  & 95.87 & 90.76 & ALL \\
%                &  & 94.26 & 85.72 & token \\ \cline{2-5}
%                 &  \multirow{2}{*}{HMM} & \textbf{91.66} & \textbf{74.9} &  \textbf{token} \\ 
%  &   & 91.54 & 69.35 &  lemma\_W0 \\ 
%%  \noalign{\smallskip}\svhline\noalign{\smallskip}
%%  & & token-level & entity-level & \\ \hline
%%\noalign{\smallskip}\hline\noalign{\smallskip}
%\end{tabular}
%\caption{Impact de la réduction des caractéristiques}
%\end{table}
%
%%Réduit de \textbf{moitié} le nombre de caractéristiques
%
%%Améliore légèrement les résultats
%
%BDS et SFFS \textcolor{red}{très lents (plus de 10 h lors de nos tests)}
%\end{frame}
%
%%\begin{frame}{Sélection des caractéristiques}
%%\tiny
%%\begin{tabular}{l|c|ccc}
%%\toprule
%%Detection Task & Tagger & {Token-level F1$^a$} & {Entity-level F1$^a$}& Features subset \\ 
%%\midrule
%%\multirow{7}{*}{Sections} 		& \multirow{4}{*}{CRF} & 99.31 & 90.48 & BDS$^{b1}$  \\
%%  				&  & \textbf{99.55} & 85.76 & \textbf{SFFS}$^{b2}$ \\
%%                &  & 99.46 & 90.03 & ALL \\
%%                &  & 91.75 & 60.26 & token \\  \cline{2-5}
%%                 & \multirow{3}{*}{HMM} & \textbf{90.99} & 3.89 & \textbf{absLength} \\ 
%% & & 86.97 & 3.65 & relLength \\   
%%  &  & 37.59 & 18.81 & token \\ \hline
%%\multirow{7}{*}{Header entities}	& \multirow{4}{*}{CRF} & 92.69 & 90.47 & BDS$^{c1}$  \\
%%				&  & \textbf{93.00} & 90.76 & \textbf{SFFS}$^{c2}$  \\ 
%%                &  & 92.74 & 90.81 & ALL \\
%%                &  & 82.73 & 72.17 & token \\ \cline{2-5}
%%                  &  \multirow{3}{*}{HMM}  & \textbf{78.61} & \textbf{56.93} &  \textbf{token} \\ 
%%  &    & 68.04 & 32.96 &  lemma\_W0 \\ 
%%  &    & 38.54 & 7.95 &  POS \\ \hline
%%\multirow{6}{*}{Norms} 			& \multirow{4}{*}{CRF} & \textbf{96.31} & 90.80 & \textbf{BDS}$^{d1}$ \\ 
%%				&  & 95.57 & 89.29 & SFFS$^{d2}$ \\ 
%%                &  & 95.87 & 90.76 & ALL \\
%%                &  & 94.26 & 85.72 & token \\ \cline{2-5}
%%                 &  \multirow{2}{*}{HMM} & \textbf{91.66} & \textbf{74.9} &  \textbf{token} \\ 
%%  &   & 91.54 & 69.35 &  lemma\_W0 \\ 
%%%  \noalign{\smallskip}\svhline\noalign{\smallskip}
%%%  & & token-level & entity-level & \\ \hline
%%\bottomrule
%%\end{tabular}
%%\end{frame}
%%
%%\begin{frame}{Sélection de la représentation de segment}
%%\tiny
%%\begin{tabular}{p{0.17\textwidth}|c|cccp{0.18\textwidth}}
%%\toprule
%%Detection Task & Tagger\hspace{0.1cm} & {Token-level F1$^a$}\hspace{0.1cm} & {Entity-level F1$^a$}\hspace{0.1cm} & training time$^b$\hspace{0.1cm} & Representation \\ 
%%\midrule
%%\multirow{8}{*}{Sections}  & \multirow{4}{*}{CRF} & 91.75 & 60.26 &  4.685  & IO \\
%%&  & 88.95 & 42.63  & 11.877 & IEO2 \\
%%&  & 87.09 & 41.45 & 12.256 & BIO2 \\
%% &  & 86.00 & 48.97  & 35.981 & BIEO \\ \cline{2-6}
%%& \multirow{4}{*}{HMM} & 32.64 & 20.41 & 6.564 & IO \\
%%&  & 32.92 & 16.87  &   7.827  & IEO2 \\
%% &  & 32.39 & 29.05 & 8.391 & BIO2 \\
%%  &  & 33.06 & 29.80 & 8.7 & BIEO \\ \hline %
%%\multirow{8}{*}{Header entities}  & \multirow{4}{*}{CRF} & 82.73 & 72.17  & 70.525 & IO \\
%% &  & 83.51 & 72.82  & 228.751 & IEO2 \\
%% &  & 82.51 & 73.49 & 230.865 & BIO2 \\
%% &  & 83.44 & 75.53 &  475.249 & BIEO \\ \cline{2-6}
%%  & \multirow{4}{*}{HMM} & 73.00 & 44.64 & 6.345 & IO \\
%%  &  & 73.40 & 50.11& 8.298 & IEO2 \\ 
%%  &  & 73.49 & 55.14 & 7.908 & BIO2 \\
%% &  & 74.12 & 60.46 & 9.973 & BIEO \\ \hline
%%\multirow{8}{*}{Norms}  & \multirow{4}{*}{CRF} & 94.26 & 85.72 & 28 & IO \\%
%%&  & 94.27 & 87.17 & 32.136 & IEO2 \\
%% &  & 94.24 & 84.37 & 50.769 & BIO2 \\
%%  &  & 93.47 & 86.72 & 50.566 & BIEO \\ \cline{2-6}
%%  & \multirow{4}{*}{HMM} & 89.30 & 74.37 &  41.389 & IO \\%  
%%   &  & 88.71 & 79.61 & 44.086 & IEO2 \\
%%  &  & 88.53 & 75.24 & 46.634 & BIO2 \\
%%  &  & 87.74 & 79.99 & 45.52& BIEO \\
%%\bottomrule
%%\end{tabular}
%%\end{frame}
%
%\begin{frame}{Nombre nécessaire de données d'entrainement}
%\begin{figure}[!h]
%\includegraphics[width=0.9\textwidth]{lc-crf.png}
%\caption{Résultats en fonction du nombre de données d'entrainement (fractions d'environ 380 décisions)}\label{p4_crf-learning-curves}
%\end{figure}
%\end{frame}
%
%\section{Extraction d'informations sur les demandes}
%
%\begin{frame}{Extraction des informations sur les demandes}
%\begin{block}{Informations pertinentes à extraire}
%\begin{itemize}
%\item \textbf{Position de la partie}: Intimé
%\item \textbf{Catégorie de demande}: Dommages-intérêts pour procédure abusive
%\begin{itemize}
%\item \textbf{Objet}: Dommages-intérêts
%\item \textbf{Fondement}: Articles 1382 code civil et 32-1 code de procédure civile
%\end{itemize}
%\item \textbf{Quantum demandé}: 20 000 euros
%\item \textbf{Résultat} : Rejet
%\item \textbf{Quantum accordé} : 0 euros
%\end{itemize}
%\end{block}
%\end{frame}
%
%\begin{frame}{Difficultés}
%%\small
%Expressions non structurées, par  \textcolor{orange}{référence}, par \textcolor{blue}{agrégation}
%
%\begin{exampleblock}{Expression de demande}
%La société A. conclut à la confirmation du jugement entrepris sauf à
%former appel incident sur la disposition du jugement l'ayant déboutée de sa
%demande de \textbf{dommages intérêts pour abus de procédure} et elle demande à la cour de
%condamner l'appelante à lui payer la somme de \textbf{20 000 euros} à titre de dommages
%intérêts ...
%
%%CATOU972015.xml
%%Aux termes de ses conclusions ..., la S.A.S. S. présente \textcolor{orange}{les mêmes demandes que celles de la S.A. V. F.}
%
%%- \textbf{la} condamner à payer une somme de \textbf{283 589 euros} à titre de {\bf dommages et intérêts pour concurrence déloyale},
%
%\end{exampleblock}
%
%\begin{exampleblock}{Expression de resultat}
%La cour, ... 
%
%Confirme \textcolor{orange}{la décision entreprise} en \textcolor{blue}{toutes ses dispositions},
%\end{exampleblock}
%\end{frame}
%
%\begin{frame}{Approche supervisée d'extraction des demandes}
%\begin{block}{Simplification du problème}
%\begin{itemize}
%\item On suppose qu'une décision ne comprend qu'au plus une demande d'une catégorie donnée
%\item Méthode générique qui s'adapte aux spécificités de la catégorie traitée
%\item Définition incrémentale des catégories
%\end{itemize}
%\end{block}
%\end{frame}
%
%%\begin{frame}{Détection d'une catégorie par classification}
%%\begin{figure}
%%\includegraphics[scale=0.6]{archi-classif.png}
%%\caption{Approche d'expérimentation de la classification}
%%\end{figure}
%%\end{frame}
%
%\begin{frame}{Approche supervisée d'extraction des demandes}
%(1) Sélection de termes caractéristiques
%\begin{exampleblock}{Dommages-interets pour abus de procedure}
%\small
%\begin{tabular}{l|c}
%\textbf{Terme (n-gram)} & \textbf{Poids global (NGL)}  \\ \hline
%\midrule
%procédure abusive & 15.710 \\ \hline
%pour procédure abusive & 15.007 \\ \hline
%pour procédure & 14.890 \\ \hline
%abusive & 13.721 \\ \hline
%intérêts pour procédure & 10.306 \\ \hline
%abus & 10.288 \\ \hline
%intérêts pour procédure abusive & 9.984 \\ \hline
%32-1 & 9.534\\ \hline
%%et intérêts pour procédure & 9.341 \\ \hline
%%dommages et intérêts pour procédure & 9.341 \\ \hline
%... & ...
%\end{tabular}
%\end{exampleblock}
%$ngl(w,c) = \frac{\sqrt{N} ((N_{w,c} N_{\overline{w},\overline{c}}) - (N_{w,\overline{c}} N_{\overline{w},c}))}{\sqrt{N_w N_{\overline{w}} N_c N_{\overline{c}}}}$ \cite{ng1997ngl}
%\end{frame}
%
%%\begin{frame}{Catégorisation semi-supervisée des décisions}
%%\begin{figure}
%%\includegraphics[scale=0.6]{archi-classif.png}
%%\caption{Approche d'expérimentation de la classification}
%%\end{figure}
%%\end{frame}
%
%\begin{frame}{ Détection d'une catégorie par classification binaire}
%
%Conditions d'expérimentation
%
%\begin{itemize}
%\small
%\item Représentation vectorielle: {\small \[ poids(w*, t) = poids_{local}(w*, t) * poids_{global}(w*) * facteur_{normalisation}\]}
%\item Évaluation de différentes configurations:
%\begin{itemize}
%\item dimensions des vecteurs : 10, ..., 250, ...
%\item méthodes de sélection de termes discriminants: $\chi^2, \Delta_{DF}, Marascuilo, NGL, GSS $  ...
%\item méthodes de classification: SVM, arbre de décision, KNN, naïf bayésien (avec Weka\cite{Eibe2016Weka})
%\item méthodes de pondération locale : TF, LogTF, ATF, TP
%\end{itemize}
%\item environ 2000 cas inconnus, 
%\item dommages-intérêts pour abus de procédure : entrainement 152 positifs, test 39 positifs + 157 négatifs
%\item prestation compensatoire : entrainement 100 positifs, test 100 positifs + 100 négatifs
%\end{itemize}
%\end{frame}
%
%\begin{frame}{Détection d'une catégorie par classification binaire}
%
%Premiers résultats:
%
%%{\small $ poids(w*, t) = poids_{local}(w*, t) * poids_{global}(w*) * facteur_{normalisation}$}
%
%\begin{figure}
%\includegraphics[width=0.75\textwidth]{f-mesure-classif.png}
%\caption{Résultats des meilleures configurations (taille des vecteurs, poids global, poids local, modèle de classifieur)}
%\end{figure}
%\end{frame}
%
%\begin{frame}{Extraction du sens du résultat (avec la même approche)}
%Classification des décisions d'une catégorie prédéfinie
%\begin{figure}
%\includegraphics[width=0.8\textwidth]{classifResultat.png}
%\caption{Résultats des meilleures configurations (taille des vecteurs, poids global, poids local, modèle de classifieur)}
%\end{figure}
%\end{frame}
%
%\begin{frame}{Extraction du sens du résultat (méthodes Gini-PLS)}
%
%Combinaison de 2 méthodes de régression:
%\begin{enumerate}
%\setlength\itemsep{1.5em}
%\item PLS: réduction supervisée des dimensions $x_1, x_2, ..., x_p$ en composantes orthogonales $t_1, ...., t_h$
%
%$t_h = w_{h1} x_1 + \cdots + w_{hj} x_j + \cdots + w_{hp} x_p$
%
%avec $w_{hj} = \frac{cov(u_{(h-1)j}, \epsilon_h)}{\sqrt{\sum_p^{j=1} cov^2(u_{(h-1)j}, \epsilon_h)}}$
%, $y=c_1 t_1 + ... + c_h t_h + \epsilon_h$,
%
%et $x_j=\beta_{1j} t_1 + ... + \beta_{hj} t_h + u_{(h-1)j}$
%
%\item Gini: élimination de la sensibilité au \textit{outliers} en remplaçant la covariance $cov(x_j, y)$ par la covariance de Gini $cog(y; x_j) := cov(y; R(x_j))$
%\end{enumerate}
%
%\cite{souissi2013gini}
%
%\end{frame}
%
%\section{Activités complémentaires}
%\begin{frame}{\currentname}
%\begin{itemize}
%\setlength\itemsep{1em}
%\item Formations complémentaires: 11 modules (132h)
%\item Enseignement: travaux pratiques (Big Data avec Hadoop)
%\item Valorisation des travaux: 
%\begin{itemize}
%\item Conférence EGC, Grenoble, janvier 2017
%\item 1 article en relecture (AKDM8)
%\item Démo des 1er résultats: SAT AXLR (Montpellier)
%\item Séminaire e-juris (Lyon)
%\end{itemize}
%\item Participation au challenge COLIEE: 4e place / 12
%\end{itemize}
%\end{frame}
%
%\section{Conclusion et plan de travail}
%
%\begin{frame}{Résumé}
%\begin{itemize}
%\item Détection d'entités et de sections basée HMM / CRF
%\begin{itemize}
%\item Bons résultats même avec un peu de données annotées
%\item Difficultés:
%\begin{itemize}
%\item Annotation manuelle d'un jeu suffisant d'exemples
%\item Identification de bons descripteurs 
%\item Lenteur de la sélection de caractéristiques
%\end{itemize}
%\item Limite de l'approche:
%\begin{itemize}
%\item Descripteurs définis manuellement 
%\item Etiquetage en plusieurs passes 
%\end{itemize}
%\end{itemize}
%\item Détection de termes propres aux catégories de demandes
%\item Détection des catégories par classification
%\item Détection moins triviale du sens du résultat 
%\end{itemize}
%
%\end{frame}
%
%\begin{frame}{Organisation du travail en 3 problématiques}
%
%\begin{enumerate}
%\setlength\itemsep{1.5em}
%\item Extraction des demandes et résultats par affinement de la segmentation des textes
%%\item Catégorisation supervisée vs. non-supervisée des demandes extraites 
%\item Standardisation et représentation des informations extraites sous forme de base de connaissances
%\item Détermination des facteurs associables aux décisions des juges (faits ou arguments)
%
%\end{enumerate}
%
%\end{frame}
%
%\section{Questions?}
%
%%\begin{frame}{Quelle aide à l' \og analyse \fg{} est précisément attendue?}
%%\begin{enumerate}
%%\setlength\itemsep{2em}
%%\item Comparaison des taux des types de résultat : 
%%
%%positif vs. négatif ...
%%\begin{itemize}
%%\setlength\itemsep{1em}
%%\item pour une catégorie définie de demandes
%%\item dans une situation factuelle d'intérêt
%%\end{itemize}
%%\item Détermination d'idées associées à un type de résultat 
%%\begin{itemize}
%%\setlength\itemsep{1em}
%%\item faits
%%\item arguments
%%\end{itemize}
%%\end{enumerate}
%%\end{frame}
%%
%%\begin{frame}{Organisation prévisionnelle de la rédaction}
%%\begin{enumerate}
%%\item Introduction (Positionnement, Attendu, défis, généralités sur les compétences techniques nécessaires)
%%\item Étude bibliographique: Analyse sémantique et prédictive des décisions judiciaires
%%\item Extraction/détection d'information à partir de texte: expression et résolution d'entités, de demandes et de résultats, liaison des demandes et résultats correspondances
%%\item Interprétation: Catégorisation des demandes et résultats
%%\item Explication : détermination des facteurs liés au sens du résultat
%%\item Conclusion: apport, voies d'amélioration, autres problématiques  
%%\end{enumerate}
%%\end{frame}

%-=-=-=-=-=-=-=-=-=-=-=-=-=-=-=-=-=-=-=-=-=-=-=-=
%	References:
%-=-=-=-=-=-=-=-=-=-=-=-=-=-=-=-=-=-=-=-=-=-=-=-=
\begin{frame}[t,allowframebreaks]{References}
\tiny
\bibliographystyle{apalike}
\bibliography{references}	
\end{frame}

\end{document}
