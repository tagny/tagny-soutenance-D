	%-=-=-=-=-=-=-=-=-=-=-=-=-=-=-=-=-=-=-=-=-=-=-=-=
%
%        LOADING DOCUMENT
%
%-=-=-=-=-=-=-=-=-=-=-=-=-=-=-=-=-=-=-=-=-=-=-=-=

\documentclass[pagenumber]{beamer}
\usetheme{sthlm}
%\usecolortheme{sthlmv42}

%-=-=-=-=-=-=-=-=-=-=-=-=-=-=-=-=-=-=-=-=-=-=-=-=
%        LOADING PACKAGES
%-=-=-=-=-=-=-=-=-=-=-=-=-=-=-=-=-=-=-=-=-=-=-=-=
\usepackage[utf8]{inputenc}
\usepackage[frenchb]{babel}
%\usepackage[normalem]{ulem}
\usepackage{roboto}
\usepackage{caption}
\captionsetup{font=scriptsize}
%\usepackage[font=footnotesize]{subcaption}
% in preamble
\usepackage{chronology}
\usepackage{pgf}
%\usepackage[linesnumbered,ruled,vlined]{algorithm2e}
\usepackage[titlenotnumbered,ruled,vlined,noend,linesnumbered,french, onelanguage]{algorithm2e}
\usepackage{tikz}
\usetikzlibrary{arrows,automata}
\usepackage{array,multirow}
\usepackage{nameref}
\usepackage{listings}
\usepackage{eurosym}
%\usepackage{subfigure}
\usepackage[font=small,singlelinecheck=off,justification=centering]{caption}
\usepackage[font=footnotesize]{subcaption}
\usepackage{xcolor}
\definecolor{shadecolor}{RGB}{51,51,51}
\definecolor{darkblue}{rgb}{0.0,0.0,0.6}

\makeatletter
\newcommand*{\currentname}{\@currentlabelname}

% Automatic frame titles and subtitles
\newcommand{\mysubsectiontitle}{\thesection.\thesubsection~\insertsubsectionhead}
\newcommand{\mysectiontitle}{\thesection~\insertsectionhead}

\makeatother

\graphicspath{ {fig/} }
% add page number
%\usepackage[defaultsans]{cantarell}

\newcommand{\p}{\mathbb{P}}
\newcommand{\setsize}[1]{{\vert#1\vert}}
\newcommand{\twodots}{\mathrel{{.}\,{.}}\nobreak}
\setbeamerfont{title}{series=\upshape}
\setbeamertemplate{footline}{\hfill\footnotesize\insertframenumber\hskip3pt\null\vskip3pt}

\newcommand{\argmax}{\mathop{\mathrm{argmax}}\limits}
\renewcommand{\max}{\mathop{\mathrm{max}}\limits}

\renewcommand{\event}[3][e]{%
  \pgfmathsetlength\xstop{(#2-\theyearstart)*\unit}%
  \ifx #1e%
    \draw[fill=black,draw=none,opacity=0.5]%
      (\xstop, 0) circle (.2\unit)%
      node[opacity=1,rotate=45,right=.2\unit] {#3};%
  \else%
    \pgfmathsetlength\xstart{(#1-\theyearstart)*\unit}%
    \draw[fill=black,draw=none,opacity=0.5,rounded corners=.1\unit]%
      (\xstart,-.1\unit) rectangle%
      node[opacity=1,rotate=45,right=.2\unit] {#3} (\xstop,.1\unit);%
  \fi}%

\addto\captionsfrench{%
\renewcommand{\figurename}{\scriptsize {\scshape Figure}}
\renewcommand{\tablename}{\scriptsize {\scshape Table}}
%\renewcommand{\ALG@name}{\tiny {\scshape Algorithme}}
}


\lstset{
	basicstyle=\tiny,
	breaklines=true,
	showstringspaces=false,
	inputencoding=utf8,
	extendedchars=false,
	literate=% {à}{{\`a}}1 {â}{{\^a}}1 {é}{{\'e}}1 {è}{{\`e}}1 {ê}{{\^e}}1 {î}{{\^i}}1 {ô}{{\^o}}1 {ù}{{\`u}}1 {û}{{\^u}}1 { ̧c}{{\c{}c}}1
	{á}{{\'a}}1 {é}{{\'e}}1 {í}{{\'i}}1 {ó}{{\'o}}1 {ú}{{\'u}}1
	{Á}{{\'A}}1 {É}{{\'E}}1 {Í}{{\'I}}1 {Ó}{{\'O}}1 {Ú}{{\'U}}1
	{à}{{\`a}}1 {è}{{\`e}}1 {ì}{{\`i}}1 {ò}{{\`o}}1 {ù}{{\`u}}1    
	{À}{{\`A}}1 {È}{{\'E}}1 {Ì}{{\`I}}1 {Ò}{{\`O}}1 {Ù}{{\`U}}1
	{ä}{{\"a}}1 {ë}{{\"e}}1 {ï}{{\"i}}1 {ö}{{\"o}}1 {ü}{{\"u}}1
	{Ä}{{\"A}}1 {Ë}{{\"E}}1 {Ï}{{\"I}}1 {Ö}{{\"O}}1 {Ü}{{\"U}}1
	{â}{{\^a}}1 {ê}{{\^e}}1 {î}{{\^i}}1 {ô}{{\^o}}1 {û}{{\^u}}1
	{Â}{{\^A}}1 {Ê}{{\^E}}1 {Î}{{\^I}}1 {Ô}{{\^O}}1 {Û}{{\^U}}1
	{Ã}{{\~A}}1 {ã}{{\~a}}1 {Õ}{{\~O}}1 {õ}{{\~o}}1
	{œ}{{\oe}}1 {Œ}{{\OE}}1 {æ}{{\ae}}1 {Æ}{{\AE}}1 {ß}{{\ss}}1
	{ű}{{\H{u}}}1 {Ű}{{\H{U}}}1 {ő}{{\H{o}}}1 {Ő}{{\H{O}}}1
	{ç}{{\c c}}1 {Ç}{{\c C}}1 {ø}{{\o}}1 {å}{{\r a}}1 {Å}{{\r A}}1
	{€}{{\euro}}1 {£}{{\pounds}}1 {«}{{\guillemotleft}}1
	{»}{{\guillemotright}}1 {ñ}{{\~n}}1 {Ñ}{{\~N}}1 {¿}{{?`}}1 {°}{{$^\circ$}}1 
}

\lstdefinelanguage{XML}
{
	morestring=[b]",
	morestring=[s]{>}{<},
	morecomment=[s]{<?}{?>},
	numbers=none,
	stringstyle=\color{black},
	identifierstyle=\color{darkblue},
	keywordstyle=\color{cyan},
	morekeywords={xmlns,version,type}% list your attributes here
}
%-=-=-=-=-=-=-=-=-=-=-=-=-=-=-=-=-=-=-=-=-=-=-=-=
%        BEAMER OPTIONS
%-=-=-=-=-=-=-=-=-=-=-=-=-=-=-=-=-=-=-=-=-=-=-=-=

%\setbeameroption{show notes}

%-=-=-=-=-=-=-=-=-=-=-=-=-=-=-=-=-=-=-=-=-=-=-=-=
%
%	PRESENTATION INFORMATION
%
%-=-=-=-=-=-=-=-=-=-=-=-=-=-=-=-=-=-=-=-=-=-=-=-=

\title{Méthodes d'analyse sémantique \\ de corpus de décisions jurisprudentielles}
\subtitle{\small Soutenance de thèse}
%\date{\small{\jobname}}
%\date{\scriptsize Début de thèse: 15 Décembre 2015}
\date{\scriptsize 24 janvier 2020}
\author{\centering Gildas TAGNY NGOMPÉ}
\institute{\scriptsize Jury: \tiny
\begin{itemize}
\item Stéphane MUSSARD, Professeur, Université de Nîmes (Directeur de thèse)
\item Jacky MONTMAIN, Professeur, IMT Mines Alès (Co-directeur de thèse)
\item Sandra BRINGAY, Professeur, Université Paul Valéry Montpellier (Rapporteur)
\item Mohand BOUGHANEM, Professeur, Université Toulouse III Paul Sabatier (Rapporteur)
\item Françoise SEYTE, Maître de Conférences (HDR), Université de Montpellier (Examinateur)
\item Fabrice MUHLENBACH,  Maître de Conférences, Université Jean Monnet de Saint-Étienne (Examinateur)
\item Guillaume ZAMBRANO, Maître de Conférences, Université de Nîmes (Encadrant de proximité)
\item Sébastien HARISPE,  Maître Assistant, IMT Mines Alès (Encadrant de proximité)
\end{itemize}}

\hypersetup{
pdfauthor = {\author{}: tagnyngompe@gmail.com},
pdfsubject = {},
pdfkeywords = {},
pdfmoddate= {D:\pdfdate},
pdfcreator = {}
}

\begin{document}
\nocite{}
%-=-=-=-=-=-=-=-=-=-=-=-=-=-=-=-=-=-=-=-=-=-=-=-=
%
%	TITLE PAGE
%
%-=-=-=-=-=-=-=-=-=-=-=-=-=-=-=-=-=-=-=-=-=-=-=-=
\begin{frame}[plain]
	\titlepage
\end{frame}
%}
%-=-=-=-=-=-=-=-=-=-=-=-=-=-=-=-=-=-=-=-=-=-=-=-=
%
%	TABLE OF CONTENTS: Plan
%
%-=-=-=-=-=-=-=-=-=-=-=-=-=-=-=-=-=-=-=-=-=-=-=-=
%\section*{Plan}
\begin{frame}[c]{Plan}
\tableofcontents[hideallsubsections]
\end{frame}

\section{Introduction}

\subsection{Contexte}
\begin{frame}[c]{\mysubsectiontitle}
	Motivations
	\begin{itemize}
		\item La jurisprudence analysée par les juristes pour comprendre l'application de la loi 
	    \item Difficultés de l'analyse manuelle
	    \begin{enumerate}
	    	\item Existence d'un gros volume de décisions 
	    		\includegraphics[width=0.6\textwidth]{chiffres-justice.pdf}   	
			\item Les moteurs de recherche juridique limitées : 
			\begin{itemize}
				\item Pas de critère de recherche sémantique (catégorie de demande, type de faits, etc.)
				\item pas d'analyse synthétique de corpus
			\end{itemize}
	    \end{enumerate}
	\end{itemize}
\end{frame}

%\subsection{État de l'art}
\begin{frame}[t]{\mysubsectiontitle}
	Activités en analyse automatique de décisions judiciaires	
	\begin{itemize}\scriptsize
		\item Extraction d'information dans les décisions
		\begin{itemize}  \scriptsize
			\item entités juridiques \cite{Waltl2016lexia, andrew2018legalNerAndRelation}
			\item faits \cite{wyner2010extractlegalelts, wyner2010casefactors, Shulayeva2017recognfactprincip}
			\item définitions de concept juridiques \cite{Waltl2016lexia,waltl2017legaliegerman}
			\item arguments \cite{moens2007NBvsMaxent4arguments}
		\end{itemize}
		\item Classification de décisions
		\begin{itemize} \scriptsize
			\item Prédiction des décisions de justice \cite{Ashley2009classifCases, Aletras2016predictDecisionECHR}
			\item identification de la formation et la période \cite{Sulea2017predictareadecision,sulea2017legalEnsSVM}
			\item identifier la sentence prononcée (Chine) \cite{ma2018wmdchinesecase}
		\end{itemize}
		\item Similarité entre décisions 
		\begin{itemize}  \scriptsize
			\item décisions qui citent les mêmes lois et précédents \cite{nair2018judgsimassorule}
			\item recherche d'affaires antérieures pertinentes  \cite{thenmozhi2017legalprecedretriev}
			\item identifier la sentence prononcée (Chine) \cite{ma2018wmdchinesecase}
			\item similarité basée sur la question discutée et les faits sous-jacents (Inde) \cite{kumar2011judgmentsimilarity}
			\item regroupement non-supervisé \cite{raghuveer2012legalclusteringLDA}
		\end{itemize}
	\end{itemize}
\end{frame}

\subsection{Objectif de la thèse}
\begin{frame}[c]{\mysubsectiontitle}
	Tâches et exemples d'applications
		
	\includegraphics[width=\textwidth]{Objectif_these.pdf}	
\end{frame}

\begin{frame}[c]{\mysubsectiontitle}	
	Formulation en analyse de données textuelles	\includegraphics[width=\textwidth]{Objectif_these-problemes2.pdf}		
\end{frame}


\section{Annotation des sections et entités judiciaires}

\subsection{Objectif de la tâche}
\begin{frame}[t]{\mysubsectiontitle}
	Détecter les méta-données de référence et les normes utilisées
\begin{figure}
	\includegraphics[height=\textheight]{decision-marquee.png}
\end{figure}
\end{frame}

\begin{frame}[t]{\mysubsectiontitle}
Sectionner pour organiser l'extraction des connaissances
\tiny
\begin{columns}\tiny
	\begin{column}{.45\linewidth}
		\fbox{\begin{minipage}{\textwidth}ARRÊT N°
				
				R.G: \textcolor{red}{11/03924}
				
				\textcolor{red}{COUR D'APPEL} DE \textcolor{red}{NÎMES}
				
				\textcolor{red}{CHAMBRE CIVILE}
				
				\textcolor{red}{1ère Chambre A}
				
				ARRÊT DU \textcolor{red}{20 MARS 2012}
				
				APPELANTE:
				
				\textcolor{red}{Madame Michéle A.} ...
				
				assistée de la \textcolor{red}{SELARL VAJOU}, ...
				
				INTIMES:
				
				\textcolor{red}{Monsieur Martial B} ...
				
				assisté de la \textcolor{red}{SCP MARION GUIZARD PATRICIA SERVAIS}, ...
				
				COMPOSITION DE LA COUR LORS DU DÉLIBÉRÉ:
				
				\textcolor{red}{M. Dominique BRUZY, Président}
				
				\textcolor{red}{M. Serge BERTHET, Conseiller}
				
				...
		\end{minipage}}
		\vspace{0.1cm}
		
		{\textbf{Entêtes}: méta-données}
	\end{column}
	\begin{column}{.55\linewidth}
		\fbox{\begin{minipage}{\textwidth}FAITS, PROCEDURE, ...
				
				Madame Michèle A. demande:
				
				...
				
				- de condamner Madame JONES-B. à lui payer la somme de \textcolor{red}{2.500 euros} au titre de l'\textcolor{red}{article 700 du Code de Procédure Civile}, 
		\end{minipage}}
		\vspace{0.1cm}
		
		{\textbf{Corps}: demandes, arguments et normes }
		
		\vspace{0.4cm}
		
		\fbox{\begin{minipage}{\textwidth}PAR CES MOTIFS, LA COUR:
				
				...
				
				Vu l'\textcolor{red}{article 809 du Code de Procédure Civile},
				
				...
				
				\textcolor{red}{Déboute Madame A. de sa demande de provision sur dommages-intérêts.}
				
				...
				
				Vu l'\textcolor{red}{article 700 du Code de Procédure Civile},
				
				Condamne Madame JONES-B. à verser à Madame A. la somme de \textcolor{red}{2.500 euros}.
		\end{minipage}}
		\vspace{0.1cm}
		
		{\textbf{Dispositif}: résultats et normes}
		
	\end{column}
\end{columns}
\end{frame}

\subsection{Approches probabilistes de détection d'entités}
\begin{frame}[t]{\mysubsectiontitle}
Modèles probabilistes à états et observations

\scriptsize
\begin{table}[]%width=\linewidth
%\begin{tabular}[]{p{0.40\linewidth}|p{0.55\linewidth}}
\begin{tabular}[]{c|c}
\toprule
{\textbf{HMM}} & {\textbf{CRF}} \\
%		\midrule
%		\textbf{Generative} models 	& \multicolumn{2}{c}{"\textbf{Discriminative}" or "\textbf{Conditional}" models } \\[0.25em]
%\midrule		
%"\textbf{generate}" input	& {"\textbf{condition}" on input }\\%[0.25em]
\midrule
{un seul descripteur  par observation}	& {plusieurs descripteurs complexes par observation}\\%[0.25em]
\midrule	
\begin{tikzpicture}[->,>=stealth',shorten >=1pt,auto,node distance=1.3cm,
semithick]
\node[state] (S1)                    {$s_{t-1}$};
\node[state]         (S2) [right of=S1] 	  {$s_{t}$};
\node[state]         (O) [below of=S2] {$o_{t}$};
\path (S1) edge              node {} (S2)
(S2) edge              node {} (O);
\end{tikzpicture}
& 

\begin{tikzpicture}[auto,>=stealth',shorten >=1pt,auto,node distance=1.3cm,
semithick]
\node[state] (S1)                    {$s_{t-1}$};
\node[state]         (S2) [right of=S1] 	  {$s_{t}$};
\node[state]         (O) [below of=S2] {$o_{t}$};
\path (S1) edge              node {} (S2)
(S2) edge              node {} (O);
\end{tikzpicture}					
\\%[0.25em]
\midrule
$P(S,O) = \prod\limits_{t=1}^{T} P(s_t \vert s_{t-1}) P(o_t \vert s_{t})$  & $P_\lambda(S|O) = \frac{1}{Z(O)}exp\left( \sum\limits_{t=1}^{T}\sum\limits_{k} \lambda_k f_k(s_{t-1},s_t, o_t) \right) $ \\
% & & & \\
\tiny \cite{Seymore1999hmm} & \tiny \cite{peng2006crf} \\ 
\bottomrule
\end{tabular}
\end{table}

\normalsize

Objectif: Trouver la séquence la plus probable d'étiquetage pour l'ensemble du texte

\textbf{Entrainement sur des séquences préalablement étiquetées}
\end{frame}

\subsection{Sélection de modèle}
\begin{frame}[t]{\mysubsectiontitle}		
	Méthodes explorées
	\begin{itemize} \scriptsize
		\item Descripteurs de lignes pour les sections : longueur ? position ? etc.
		\item Descripteurs de mots pour les entités : est-ce une initiale ("B.") ? est-ce un mot clé de citation de loi ? etc.
		\item Schéma d'étiquetage : distinction des parties d'une entité \tiny
		\begin{tabular}{l|ccccccccc}
			& \textit{composée} & \textit{de} & \textit{Madame} & \textit{Martine} & \textit{JEAN} & , & \textit{Président} & \textit{de} & ... \\ 
			IO & O & O & I-JUGE & I-JUGE & I-JUGE & O & I-FONCTION & I-FONCTION & ... \\
			BIO & O & O & B-JUGE & I-JUGE & I-JUGE & O & B-FONCTION & I-FONCTION & ... \\
			IEO & O & O & I-JUGE & I-JUGE & E-JUGE & O & I-FONCTION & I-FONCTION & ...\\
			BIEO & O & O & B-JUGE & I-JUGE & E-JUGE & O & B-FONCTION & I-FONCTION & ... \\
		\end{tabular}
		\item \scriptsize sélection du sous-ensemble de descripteurs court et aux meilleurs résultat (recherche par BDS et SFFS)
	\end{itemize}	
\end{frame}

\begin{frame}[t]{\mysubsectiontitle}
	Résultats (CRF)
\begin{figure}
	\begin{itemize}
		\item sélection du schéma d'étiquetage
		\begin{itemize}
			\item Les schémas plus complexes que IO rendent l'entraînement plus long
			\item Les schémas complexes ne semblent pas améliorer la détection des sections(baisse de $F_1$ de près de $20\%$)
			\item Les schémas complexes améliorent légèrement la détection d'entité de moins de $3\%$
		\end{itemize}
		\item sélection des descripteurs
		\begin{itemize}
			\item Lenteur des algorithmes BDS et SFFS (plus de 15h)
			\item BDS réduit de moitié
			\item SFFS réduit beaucoup plus
			\item Pas d'amélioration ou détérioration considérable de la détection
		\end{itemize}
	\end{itemize}
\end{figure}
\end{frame}

\subsection{Discussions des résultats}
\begin{frame}[t]{\mysubsectiontitle}
	Confusions de labels
\begin{figure}[!htb]
	\centering
	\includegraphics[width=0.6\textwidth]{confusion_matrix_entete.png} \hfill
	\includegraphics[width=0.38\textwidth]{confusion_matrix_section.png}
	%\textcolor{red}{Matrices de confusion}
	\caption{Matrice de confusion des modèles basés CRF}
	\label{fig:structuration:matrice-confusion-entete}
\end{figure}
\end{frame}

\begin{frame}[t]{\mysubsectiontitle}
	Impact de la quantité de décisions d'entraînement
\begin{figure}[!h]
\includegraphics[width=0.9\textwidth]{lc-crf.png}
\caption{Evolution de la F1-mesure en fonction de la fraction utilisée}\label{p4_crf-learning-curves}
\end{figure}
\end{frame}

\begin{frame}[t]{\mysubsectiontitle}
	Description manuelle vs. représentation apprise
\scriptsize
\begin{tabular}{|l|l|l|l|l|l|l|}
	\hline
	&               \multicolumn{3}{c}{\textbf{CRF + descripteurs manuels}} & \multicolumn{3}{|c|}{\textbf{BiLSTM-CRF}}   \\ \cline{2-7}
	& \textit{Precision} & \textit{Rappel}                     & $F_1$ & \textit{Precision} & \textit{Rappel}      & $F_1$ \\ \hline
	\textbf{appelant}      & 82.49              & 69.42                               & 74.72       & 80.26              & 71.53                & 75.04       \\ 
	\textbf{avocat}        & 90.15              & 89.02                               & 89.56       & 84.93              & 87.88                & 86.36       \\ 
	\textbf{date}          & 95.34              & 91.46                               & 93.12       & 95.04              & 90.79                & 92.63       \\ 
	\textbf{fonction}      & 95.87              & 95.08                               & 95.44       & 92.69              & 93.48                & 93.03       \\ 
	\textbf{formation}     & 96.91              & 91.31                               & 93.7        & 91.05              & 89.47                & 89.84       \\ 
	\textbf{intervenant}   & 51.42              & 32.71                               & 36.8        & 31.48              & 20                   & 23.11       \\ 
	\textbf{intime}        & 76.01              & 79.15                               & 77.22       & 67.7               & 75.43                & 70.83       \\ 
	\textbf{juge}          & 95.67              & 94.07                               & 94.84       & 95.44              & 95.56                & 95.46       \\ 
	\textbf{juridiction}   & 98.55              & 98.25                               & 98.33       & 97.95              & 99.22                & 98.57       \\ 
	\textbf{rg}            & 95.46              & 95.29                               & 95.27       & 91.13              & 97.26                & 93.92       \\ 
	\textbf{ville}         & 98.33              & 93.01                               & 94.71       & 91.43              & 95.34                & 93.3        \\ 
	\textbf{norme}         & 91.08              & 90.27                               & 90.67       & 91.43              & 92.65                & 92.03       \\ \hline
	\noalign{\smallskip}\hline\noalign{\smallskip}
	\textbf{Evaluation globale} & 92.2               & 90.09                               & 91.12       & 89.21              & 90.43                & 89.81       \\ \hline
\end{tabular}
\end{frame}

\section{Identification des demandes}
\subsection{Objectif : identifier les informations sur les demandes}
\begin{frame}[c]{Objectif : identifier les informations sur les demandes}
Cibles : catégorie (objet+norme), sens du résultat, montant demandé, montant accordé
\begin{exampleblock}{Expression de demande et résultat}
	%danais/CASAI1401082.xml
	\scriptsize
	Jennifer M. et Catherine M. ... demandent à la Cour de :
	
	- \textcolor{orange}{infirmer le dit jugement} en \textcolor{blue}{toutes ses dispositions} ; 
	...
	
	Statuant à nouveau ...
	
	- les condamner au paiement d' une somme de  \textbf{3 000,00 € pour procédure abusive} et
	aux entiers dépens ; ...
	
	La cour ...  
	
	CONFIRME \textcolor{orange}{le jugement entrepris} en \textcolor{blue}{toutes ses dispositions}.
	
\end{exampleblock}

\tiny{\textit{Légende:  \textcolor{orange}{référence au jugement antérieur},  \textcolor{blue}{agrégation}}}


\begin{table} 
	\centering \includegraphics[width=\textwidth]{tab-danais.png}
	\caption{\scriptsize Informations à extraire (dommages-intérêts pour procédure abusive)}
\end{table}
\end{frame}

\subsection{Méthode : identifier les passages, puis les informations}

\begin{frame}[c]{Retrouver les demandes à l'aide des termes clés}
		\fbox{\parbox{\textwidth}{
				" ... 
				
				- débouter M. S. de ... % l' ensemble de ses demandes
				
				- le \underline{condamner} à payer une \textbf{amende civile} de  \textit{1.500 euros} \textbf{pour procédure abusive} ...
				
				- le \underline{condamner} à payer la somme ..."
		}}
\end{frame}
\subsection{Expérimentations sur 6 catégories de demandes}
\begin{frame}{Données}
	\begin{figure}[!htb]
		\includegraphics[width=0.8\textwidth]{chartDataset.png}
		\caption{\tiny Répartitions des demandes dans les documents annotées.}\label{fig:quanta:hist-repartition-docs}
	\end{figure}	
\end{frame}
\section{Identification du sens du résultat}
\begin{frame}[c]{Extendre le Gini-PLS pour la classification de textes}
	\only<1>{1. Méthodes}
	\only<2>{2. Résultats}		
\end{frame}
\section{Découverte des circonstances factuelles}
\begin{frame}[c]{Apprendre la similarité par transformation de document}
	\only<1>{1. Méthodes}
	\only<2>{2. Résultats}		
\end{frame}

\section{Conclusions}
\subsection{Bilan}
\begin{frame}[c]{\mysubsectiontitle}	
	\begin{itemize} \small
		\item Définition de tâches importantes pour l'analyse de corpus de décisions 
		\begin{itemize} \scriptsize
			\item Formulation en problèmes d'analyse de données textuelles
			\item Production avec un expert de données annotées d'apprentissage 
		\end{itemize}
		\item Proposition et évaluation d'approches d'extraction de connaissances jurisprudentielles :
		\begin{itemize} \scriptsize
			\item Application du HMM  et CRF pour détecter les sections et les entités juridiques \cite{tagnyngompe2017neregc, tagnyngompe2019ner}
			\item Approche d'identification des demandes par catégorie basée sur la proximité entre des termes-clés appris et les sommes d'argent		
			\item Deux extensions du Gini-PLS pour identifier le sens du résultat (article en préparation pour \textit{\textbf{Stats}})
			\item Approche d'apprentissage d'une distance de similarité pour regrouper les décisions suivant les circonstances factuelles.	
		\end{itemize}
	   \item Démonstration d'applications en analyse descriptive sur un grand corpus de décisions	   
\end{itemize}
\end{frame}

\subsection{Perspectives}
\begin{frame}[c]{\mysubsectiontitle}

	\begin{itemize} \small
		\item Extensions des propositions 
		\begin{itemize}  \scriptsize
			\item Désambiguïsation les entités détectées pour indexer les décisions
			\item Expérimentation d'approches d'extraction des évènements et relations pour l'identification des demandes		
			\item Découverte des circonstances factuelles par  {modélisation thématique}
		\end{itemize}
		\item Applications
		\begin{itemize}  \scriptsize
			\item {Anonymisation des décisions} : confidentialité des informations
			\item {Analyse prédictive} : identifier les raisons qui poussent les juges à accepter une demande 
		\end{itemize}
	\end{itemize}
\end{frame}

%\section*{Questions}
\begin{frame}
\begin{center}
	Questions
\end{center}
\end{frame}

%\usebackgroundtemplate{\includegraphics[height=\paperheight]{tribunal.png}}
%\section{\noindent\colorbox{shadecolor}
%	{\parbox{\dimexpr\textwidth-2\fboxsep\relax}{\textcolor{white}{Questions}}}}
%\usebackgroundtemplate{}

%-=-=-=-=-=-=-=-=-=-=-=-=-=-=-=-=-=-=-=-=-=-=-=-=
%	References:
%-=-=-=-=-=-=-=-=-=-=-=-=-=-=-=-=-=-=-=-=-=-=-=-=
\begin{frame}[t,allowframebreaks]{References}
\tiny
\bibliographystyle{apalike}
\bibliography{references}	
\end{frame}

\end{document}
